\chapter{Implementation and Methodology}

The implementation of the DeepMD protocol employed in this thesis is based on
the DeepMD-kit software suite \cite{Wang_DeePMD-kit_A_deep_2018}. DeepMD-kit
uses TensorFlow \cite{tensorflow2015-whitepaper} to provide an efficient and
flexible toolkit for utilizing the DeepMD scheme. DeepMD-kit comes with an
interface to the LAMMPS classical molecular dynamics code \cite{LAMMPS},
which was used in this work for computing all the material properties with the
trained machine learning models. The models were trained on MD data generated
with OpenMX, a DFT-based nanoscale material simulations software
\footnote{
    See \url{https://www.openmx-square.org/}.
}. The Python programming language
\footnote{
    See \url{https://www.python.org/}.
} alongside with its multitude of scientific computation packages was used for
building the training and evaluation infrastructure. All of the code needed to
reproduce the results presented in this thesis is available online
\footnote{
    See \url{https://github.com/hacksparr0w/bachelor-thesis}
}.

\section{Preparing the DFT Datasets}

As already stated, all DFT trajectories were calculated using OpenMX. Since
the computation resources available for this work were quite limited, we
choose to only work with simple systems of carbon, as they do not exhibit any
complicated quantum mechanical effects and thus should be easier to model and
train on. To see how well the models perform on wide variety of learning data,
three distinct datasets were prepared -- an amorphous carbon dataset, a
crystalline carbon dataset, and a mixed dataset containing both the amorphous
and the crystalline frames. Each dataset contains a selection of frames taken
from OpenMX simulations of carbon systems with distinct volumes. After the
simulation frames were generated, they were further distilled down as to not
include frames that were too similar to each other. This was done simply by
removing frames that were too close to each other in terms of their time step. 
The detailed parameters for each used dataset are listed in table
\ref{tab:datasets}.

%%
%% TODO: Elaborate on the carbon system parameters (no. atoms, box size, etc.)
%%

\begin{table}
  \begin{tabularx}{\textwidth}{lllll}
    \toprule
    \multicolumn{1}{c}{} & \multicolumn{2}{c}{Training} & \multicolumn{2}{c}{Validation} \\
    Dataset & Samples & Frames & Samples & Frames \\
    \midrule
    Crystalline & c-0.99 & 200 & c-1.01 & 200 \\
     & c-1.03 & 200 & & \\
    \midrule
    Amorphous & a-2.15 & 150 & a-2.25 & 150 \\
     & a-2.20 & 150 & a-2.31 & 150 \\
     & a-2.29 & 150 & & \\
     & a-2.30 & 150 & & \\
     & a-2.35 & 150 & & \\
     & a-2.40 & 150 & & \\
    \midrule
    Combined & c-0.99 & 200 & c-1.01 & 200 \\
     & c-1.03 & 200 & & \\
     & a-2.15 & 150 & a-2.25 & 150 \\
     & a-2.20 & 150 & a-2.31 & 150 \\
     & a-2.29 & 150 & & \\
     & a-2.30 & 150 & & \\
     & a-2.35 & 150 & & \\
     & a-2.40 & 150 & & \\
    \bottomrule
  \end{tabularx}
  \caption{The parameters used for the DFT datasets.}
  \label{tab:datasets}
\end{table}

%%
%% TODO: Include relevant OpenMX configuration here.
%%

\section{Choosing the DNN Hyperparameters}
