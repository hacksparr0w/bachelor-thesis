\chapter{Equations of State}

Equations of state (EOS) represent thermodynamic relationships between state
variables that characterize the condition of matter under specific physical
parameters, including pressure, volume, temperature, and internal energy.
State equations usually take the form of
\begin{equation}
  f(p, V, T) = 0,
\end{equation}
where $p$ is the pressure, $V$ is the volume, and $T$ the temperature of the
system. There exists no single equation of state that would accurately
predict the properties of all systems under all possible conditions. As a
consequence, a vide variery of state equations have been developed over the
years, each modelling a specific class of materials under specific conditions.
When modelling a specific system using a particular state equation, it is
important to carefully consider the limits of its applicability and its
suitability for the system of interest.

\subsection{Birch--Murnaghan Equation of State}

The Birch-Murnaghan EOS is an empirical isotropic equation that describes the
relationship between the volume, pressure, and energy of a solid material
under compression. Developed by Francis Birch in 1947, it is based on the
Murnaghan equation of state and is widely used in solid-state physics,
materials science, and geophysics to examine materials under high pressure
conditions.

The derivation of the Birch--Murnaghan EOS is based on the Eulerian finite
strain theory. When strains are infinitesimal, the dimensions of a body
decrease linearly with pressure. However, this is not the case when strains
are finite because matter becomes increasingly incompressible with pressure.
The Eulerian finite elastic strain theory treats this phenomenon. When
introducing the Eulerian scheme, it is important to note the following points:
\begin{itemize}
  \item Unlike the Lagrangian scheme, which uses the pre-compression state as
  a reference, the Eulerian scheme uses the post-compression state.
  \item Changes are expanded in squared length before and after compression.
\end{itemize}

We start by considering a cube of side $x_0$ and volume $V_0 = x_0^3$. We
apply a compression $u$ ($u < 0$), so that the length of the side becomes
$x = x_0 + u$. The change in squared length is expressed as
\begin{equation}
  x^2 - x_0^2 = x^2 - (x - u)^2 = 2ux - u^2.
\end{equation}
The displacement $u$ under uniform compression should be proportional to the
length of the cube's side with a proportionality constant $c$
\begin{equation}
  u = cx.
\end{equation}
The constant $c$ is referred to as the strain in the linear elasticity.
Combining the above equations we get
\begin{equation}
  x^2 - x_0^2 = 2cx^2 - c^2x^2 = 2x^2(c - \frac{1}{2}c^2).
\end{equation}
We obtain the definition for the second-power Eulerian finite strain
\begin{equation}
  \varepsilon_{E2} = c - \frac{1}{2}c^2.
\end{equation}
We can use the Eulerian finite strain to express the change in squared length
of the side of the cube as
\begin{equation}
  x^2 - x_0^2 = 2\varepsilon_{E2}x^2.
\end{equation}
The ration of the pre-compression side length to the post-compression side
length can be expressed using the Eulerian finite strain as
\begin{equation}
  \frac{x_0}{x} = \sqrt{1 - 2\varepsilon_{E2}}.
\end{equation}
We can use this equation to also express the ratio of the pre-compression
volume and the post-compression volume as
\begin{equation}
  \frac{V_0}{V} = \left(\frac{x_0}{x}\right)^3 =
  \left(1 - 2 \varepsilon_{E2}\right)^{\frac{3}{2}}.
\end{equation}
Conversely, we can express the finite strain using the volume ratio
\begin{equation}
  \varepsilon_{E2} = \frac{1}{2} \left[
    1 - \left(\frac{V_0}{V}\right)^{\frac{3}{2}}
  \right].
\end{equation}
The $\varepsilon_{E2}$ when the body is compressed. To make the strain
positive, we define $f_{E2}$ as
\begin{equation}
  f_{E2} = -\varepsilon_{E2} = \frac{1}{2} \left[
    \left(\frac{V_0}{V}\right)^{\frac{3}{2}} - 1
  \right].
  \label{eq:finite-strain}
\end{equation}
The first law of thermodynamics in terms of Helmholtz free energy is given by
\begin{equation}
  \mathrm{d}F = - S \mathrm{d}T - p \mathrm{d}V.
\end{equation}
We can simply derive an expression for pressure of isothermal system as
\begin{equation}
  p = - \left(\frac{\partial F}{\partial V}\right)_T.
  \label{eq:isothermal-pressure}
\end{equation}
The Helmholtz free energy of matter may be expressed by a series of the
Eulerian finite strain
\begin{equation}
  F = a_0 + a_1 f_{E2} + a_2 f_{E2}^2 + a_3 f_{E2}^3 + \cdots.
\end{equation}
The absolute value of $F$ is abitrary, so we can set $a_0 = 0$. Pressure
should be zero in an uncompressed state, as should $f_{E2}$, so from the
equation \eqref{eq:isothermal-pressure} we get
\begin{equation}
  p_{f_{E2}=0} = - \left(
    \frac{\partial F}{\partial f_{E2}}
  \right)_{T,f_{E2}=0} \left(
    \frac{\partial f_{E2}}{\partial V}
  \right)_{T, f_{E2}=0} = -a_1 \left(
    \frac{\partial f_{E2}}{\partial V}
  \right)_{T, f_{E2}=0} = 0.
\end{equation}
Therefore coefficient $a_1$ in the expansion of $F$ is $a_1 = 0$. We continue
by truncating the expansion to
\begin{equation}
  F \doteq a_2 f_{E2}^2 + a_3 f_{E2}^3.
\end{equation}
Plugging this into the equation \eqref{eq:isothermal-pressure}, we get
\begin{equation}
\begin{alignedat}{2}
  p &= -\left[
    \frac{\partial}{\partial V}(a_2 f_{E2}^2 + a_3 f_{E2}^3)
  \right]_T \\
  &= - (2 a_2 f_{E2} + 3 a_3 f_{E2}^2)
  \left( \frac{\partial f_{E2}}{\partial V}\right)_T \\
  &= -2 a_2 (1 + \xi_1 f_{E2}) f_{E2} \left(
    \frac{\partial f_{E2}}{\partial V}
  \right)_T,
  \label{eq:pressure}
\end{alignedat}
\end{equation}
where $\xi_1 = 3 a_3/2 a_2$. As derived in \cite{min9120745}, the parameter
$\xi_1$ is given by
\begin{equation}
  \xi_1 = \frac{3}{2}\left(B_{T_0}' - 4\right),
  \label{eqn:xi_1}
\end{equation}
where $B_{T_0}'$ is the pressure derivative of the isothermal bulk modulus at
standard temperature. Refferring to the same paper, the coefficient $a_2$ is
\begin{equation}
  a_2 = \frac{9}{2} B_{T_0} V_0,
  \label{eqn:a_2}
\end{equation}
where $B_{T_0}$ is the isothermal bulk modulus at standard temperature. From
definition of finite strain we get
\begin{equation}
  \left(\frac{\partial f_{E2}}{\partial V}\right)_T = - \frac{1}{3 V_0} \left(
    \frac{V_0}{V}
  \right)^{\frac{5}{3}}.
  \label{eq:finite-strain-derivative}
\end{equation}
By plugging equations \eqref{eq:finite-strain},
\eqref{eq:finite-strain-derivative}, \eqref{eqn:a_2}, \eqref{eqn:xi_1} into
\eqref{eq:pressure}, we recover the the third-order Birch--Murnaghan EOS of
the form
\begin{equation}
  \small{p = \frac{3}{2} K_{T_0} \left[
    \left(\frac{V_0}{V}\right)^{\frac{7}{3}}
    - \left(\frac{V_0}{V}\right)^{\frac{5}{3}}
  \right] \left[
    1 + \frac{3}{4}(B_{T_0}' - 4)\left\{
      \left(\frac{V_0}{V}\right)^{\frac{2}{3}} - 1
    \right\}
  \right].}
\end{equation}
