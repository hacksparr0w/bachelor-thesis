\chapter{Conclusion}

This work used the DeepMD methodology to demonstrate the viability and
efficiency of machine learning potentials for performing molecular dynamics
and calculating material properties of both crystalline and amorphous
materials. We've trained wide variety of DNN models to show how the
hyperparameter selection affects the inference accuracy and computational
efficiency. We then used the best performing models to perform molecular
dynamics simulations of crystalline and amorphous silicon. Subsequently, we
used the data from the molecular dynamics simulations to compute the
relaxation volume $V_0$, the bulk modulus $B_0$ and the volumetric thermal
expansion coefficient $\alpha_V$ of silicon. The results from the machine
learning models were very promising and showed that we can get quite accurate
results with relative errors of $\sim 0.3\%$ for the relaxation volume $V_0$
of crystalline silicon, $\sim 17 \%$ for the bulk modulus $B_0$ of crystalline
silicon, $\sim 10 \%$ for the bulk modulus $B_0$ of amorphous silicon. The
models had some trouble with the volumetric thermal expansion coefficient
$\alpha_V$ whose relative error was $\sim 290 \%$ but we believe that this was
largerly due to the shortcomings of our methodology caused mostly by the lack
of computational resources and time and not by the inherent conceptual flaws
in the DNN modeling approach. We note that the accuracy of the results
can be improved drastically by using larger training data sets and that the
quality and size of the training data set is the most important factor
limiting the model prediction accuracy. We also note that this may be quite
problematic as quality training data can only be obtained with computationally
expensive quantum mechanical simulations and thus is the main bottleneck for
the DeepMD methodology.

Future work should focus on improving the quality of the training data sets,
coming up with methodologies to generate training data on the fly (e.g. by
using the active learning approach), comparing the DeepMD methodology with
an alternative machine learning implementations or enhancing the current
DeepMD protocol presented in this work by moving to GPU-accelerated learning
and inference. 
