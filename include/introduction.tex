\chapter{Introduction}

Molecular dynamics (MD) is a computer simulation method for calculating the
physical movements of atoms and molecules. MD simulations work by numerically
solving Newton's equations to obtain the trajectories of atoms and molecules.
The forces acting on the simulated particles, along with their potential
energies, are calculated using interatomic potentials or molecular mechanical
force fields. MD is an invaluable tool used in many disciplines, including
physics, chemistry, biology, and material science, as it enables running
complex experiments virtually, examining their results on atomary scales,
which might be unreachable even with costly laboratory equipment.

The accuracy of MD simulations is largely dependent on its model for atomic
interactions. There are two prominent approaches to modeling these
interactions. The \textit{ab initio} calculations determine the electronic
structure of a system from first principles using a quantum mechanical theory.
\textit{Ab initio} molecular dynamics (AIMD) is considered to be highly
accurate, however, due to treating the electronic degrees of freedom, its
computational burden is significant, and its usage is generally limited to
smaller systems with hundreads of atoms and time scales of
$\pm 100 \, \mathrm{ps}$. Applications requiring larger cells and longer
simulations are currently accessible only with empirical interatomic
potentials, also called force fields (FFs). These models, while having much
lower computational requirements, suffer from decreased precision and
are not very transferable.

In recent years, there's been a steady surge of breakthrough applications of
machine learning (ML) and neural networks. From beating human experts in
games like chess and go to almost perfectly solving the problem of protein
folding, artificial intelligence seems to be an ideal fit for finding
stochastic heuristics and patterns in complicated problems, given a large
dataset of learning data. This wave of transition toward machine
learning-based algorithms has affected even MD software packages and given
rise to a very successful class of machine learning potentials and force
fields.

The goal of my thesis is to evaluate these machine learning potentials and
demonstrate that they are capable of producing results that are on par with
quantum mechanical methods while scaling in a linear manner with the size of a
system. My application uses the DeePMD-kit code, which is an implementation of
the Deep Potential Molecular Dynamics (DeePMD) protocol. Furthermore, the
LAMMPS Molecular Dynamics Simulator was used to run the machine learning
potentials on concrete molecular dynamics ensembles. Last but not least, the
OpenMX simulator was used to provide quantum mechanical data for training and
evaluating the neural network models.

%
% TODO: Briefly sum-up contents of individual chapters
%
