\documentclass[
  digital,
  color,
  oneside,
  nosansbold,
  nocolorbold,
  lof,
  lot
]{fithesis4}
\usepackage[resetfonts]{cmap}
\usepackage[T1,T2A]{fontenc}
\usepackage[
  main=english,
  english, german, russian, czech, slovak %% The additional keys allow
]{babel}
\usepackage{paratype}
\def\textrussian#1{{\usefont{T2A}{PTSerif-TLF}{m}{rm}#1}}
\thesissetup{
    date        = \the\year/\the\month/\the\day,
    university  = mu,
    faculty     = sci,
    type        = bc,
    department  = Department of Physical Electronics,
    author      = Tomáš Rottenberg,
    gender      = m,
    programme   = Nanotechnology,
    field       = Physics,
    advisor     = {Mgr. Pavel Ondračka, Ph.D.},
    title       = {Machine learning-assisted atomistic modeling of amorphous materials},
    TeXtitle    = {Machine learning-assisted atomistic modeling of amorphous materials},
    keywords    = {machine learning, computational physics, atomistic modeling},
    TeXkeywords = {machine learning, computational physics, atomistic modeling},
    abstract    = {%
      This is the abstract of my thesis, which can

      span multiple paragraphs.
    },
    thanks      = {%
      These are the acknowledgements for my thesis, which can

      span multiple paragraphs.
    },
    bib         = example.bib,
    facultyLogo = fithesis-sci,
}
\usepackage{makeidx}      %% The `makeidx` package contains
\makeindex                %% helper commands for index typesetting.
\usepackage[acronym]{glossaries}          %% The `glossaries` package
\renewcommand*\glspostdescription{\hfill} %% contains helper commands
\loadglsentries{example-terms-abbrs.tex}  %% for typesetting glossaries
\makenoidxglossaries                      %% and lists of abbreviations.
%% These additional packages are used within the document:
\usepackage{paralist} %% Compact list environments
\usepackage{amsmath}  %% Mathematics
\usepackage{amsthm}
\usepackage{amsfonts}
\usepackage{url}      %% Hyperlinks
\usepackage{markdown} %% Lightweight markup
\usepackage{listings} %% Source code highlighting
\lstset{
  basicstyle      = \ttfamily,
  identifierstyle = \color{black},
  keywordstyle    = \color{blue},
  keywordstyle    = {[2]\color{cyan}},
  keywordstyle    = {[3]\color{olive}},
  stringstyle     = \color{teal},
  commentstyle    = \itshape\color{magenta},
  breaklines      = true,
}
\usepackage{floatrow} %% Putting captions above tables
\floatsetup[table]{capposition=top}
\usepackage[babel]{csquotes} %% Context-sensitive quotation marks
\begin{document}
%% Uncomment the following lines (by removing the %% at the beginning)
%% and to print out List of Abbreviations and/or Glossary in your
%% document. Titles for these tables can be changed by replacing the
%% titles `Abbreviations` and `Glossary`, respectively.
%% \clearpage
%% \printnoidxglossary[title={Abbreviations}, type=\acronymtype]
%% \printnoidxglossary[title={Glossary}]

%% The \chapter* command can be used to produce unnumbered chapters:
\chapter*{Introduction}
%% Unlike \chapter, \chapter* does not update the headings and does not
%% enter the chapter to the table of contents. I we want correct
%% headings and a table of contents entry, we must add them manually:
\markright{\textsc{Introduction}}
\addcontentsline{toc}{chapter}{Introduction}

Theses are rumoured to be \enquote{the capstones of education}, so
I decided to write one of my own. If all goes well, I will soon
have a diploma under my belt. Wish me luck!

\chapter{Using lightweight markup}
\shorthandoff{-}
\begin{markdown*}{%
  hybrid,
  definitionLists,
  footnotes,
  inlineFootnotes,
  hashEnumerators,
  fencedCode,
  citations,
  citationNbsps,
  pipeTables,
  tableCaptions,
}

If you decide that \LaTeX{} is too wordy for some parts of your
document, there are [packages](https://www.ctan.org/pkg/markdown
"Markdown") that allow you to use more lightweight markup next
to it.

 ![logo](fithesis/logo/mu/fithesis-fi-color "The old-style seal of
         the Faculty of Informatics")

| Right | Left | Default | Center |
|------:|:-----|---------|:------:|
|    12 | 12   | 12      |   12   |
|   123 | 123  | 123     |   123  |
|     1 | 1    | 1       |    1   |

: This is a table with different types of alignment.

This is a bullet list. Unlike numbered lists, bulleted lists
contain an **unordered** set of bullet points. When a bullet point
contains multiple paragraphs, the list is typeset as follows:

  * The first item of a bullet list

    that spans several paragraphs,
  * the second item of a bullet list,
  * the third item of a bullet list.

When none of the bullet points contains multiple paragraphs, the
list has a more compact form:

  * The first item of a bullet list,
  * the second item of a bullet list,
  * the third item of a bullet list.

Unlike a bulleted list, a numbered list implies chronology or
ordering of the bullet points. When a bullet point
contains multiple paragraphs, the list is typeset as follows:

  1. The first item of an ordered list

     that spans several paragraphs,
  2. the second item of an ordered list,
  3. the third item of an ordered list.
  #. If you are feeling lazy,
  #. you can use hash enumerators as well.

When none of the bullet points contains multiple paragraphs, the
list has a more compact form:

  6. The first item of an ordered list,
  7. the second item of an ordered list,
  8. the third item of an ordered list.

Definition lists are used to provide definitions of terms. When
a definition contains multiple paragraphs, the list is typeset
as follows:

Term 1

:   Definition 1

*Term 2*

:   Definition 2

        Some code, part of Definition 2

    Third paragraph of Definition 2.

When none of the bullet points contains multiple paragraphs, the
list has a more compact form:

Term 1
:   Definition 1
*Term 2*
:   Definition 2

Block quotations are used to include an excerpt from an external
document in way that visually clearly separates the excerpt from
the rest of the work:

> This is the first level of quoting.
>
> > This is nested blockquote.
>
> Back to the first level.

Footnotes are used to include additional information to the
document that are not necessary for the understanding of the main
text. Here is a footnote reference^[Here is the footnote.] and
another.[^longnote]

[^longnote]: Here's one with multiple blocks.

    Subsequent paragraphs are indented to show that they
belong to the previous footnote.

        Some code

    The whole paragraph can be indented, or just the first
    line.  In this way, multi-paragraph footnotes work like
    multi-paragraph list items.

Citations are used to provide bibliographical references to other
documents. This is a regular citation~[@borgman03, p. 123]. This is
an in-text citation: @borgman03\. You can also cite several authors
at once using both regular~[see @borgman03, p. 123; @greenberg98,
sec.  3.2; and @thanh01] and in-text citations: @borgman03 [p.123;
@greenberg98, sec. 3.2; @thanh01].

Code blocks are used to include source code listings into the
document:

    #include <stdio.h>
    #include <unistd.h>
    #include <sys/types.h>
    #include <sys/wait.h>
    // This is a comment
    int main(int argc, char **argv)
    {
        while (--c > 1 && !fork());
        sleep(c = atoi(v[c]));
        printf("%d\n", c);
        wait(0);
        return 0;
    }

There is an alternative syntax for code blocks that allows you to
specify additional information, such as the language of the source
code. This information can be used for syntax highlighting:

``` sh
#!/bin/sh
fac() {
  if [ "$1" -leq 1 ]; then
    echo 1
  else
    echo $(("$1" * fac $(("$1" - 1))))
  fi
}
``````````````

~~~~~~ Ruby
# Here's a way to empty an array.
joe = [ 'eggs.', 'some', 'break', 'to', 'Have' ]
print(joe.pop, " ") while joe.size > 0
print "\n"
~~~~~~

\end{markdown*}
\shorthandon{-}

\chapter{These are}
\section{the available}
\subsection{sectioning}
\subsubsection{commands.}
\paragraph{Paragraphs and}
\subparagraph{subparagraphs are available as well.}
Inside the text, you can also use unnumbered lists,
\begin{itemize}
  \item such as
  \item this one
  \begin{itemize}
    \item     and they can be nested as well.
    \item[>>] You can even turn the bullets into something fancier,
    \item[\S] if you so desire.
  \end{itemize}
\end{itemize}
Numbered lists are
\begin{enumerate}
  \item very
  \begin{enumerate}
    \item similar
  \end{enumerate}
\end{enumerate}
and so are description lists:
\begin{description}
  \item[Description list]
    A list of terms with a description of each term
\end{description}
The spacing of these lists is geared towards paragraphs of text.
For lists of words and phrases, the \textsf{paralist} package
offers commands
\begin{compactitem}
  \item that
  \begin{compactitem}
    \item are
    \begin{compactitem}
      \item better
      \begin{compactitem}
        \item suited
      \end{compactitem}
    \end{compactitem}
  \end{compactitem}
\end{compactitem}
\begin{compactenum}
  \item to
  \begin{compactenum}
    \item this
    \begin{compactenum}
      \item kind of
      \begin{compactenum}
        \item content.
      \end{compactenum}
    \end{compactenum}
  \end{compactenum}
\end{compactenum}
The \textsf{amsthm} package provides the commands necessary for the
typesetting of mathematical definitions, theorems, lemmas and
proofs.

%% We will define several mathematical sectioning commands.
\newtheorem{theorem}{Theorem}[section] %% The numbering of theorems
                               %% will be reset after each section.
\newtheorem{lemma}[theorem]{Lemma}         %% The numbering of lemmas
\newtheorem{corollary}[theorem]{Corollary} %% and corollaries will
                               %% share the counter with theorems.
\theoremstyle{definition}
\newtheorem{definition}{Definition}
\theoremstyle{remark}
\newtheorem*{remark}{Remark}

\begin{theorem}
  This is a theorem that offers a profound insight into the
  mathematical sectioning commands.
\end{theorem}
\begin{theorem}[Another theorem]
  This is another theorem. Unlike the first one, this theorem has
  been endowed with a name.
\end{theorem}
\begin{lemma}
  Let us suppose that $x^2+y^2=z^2$. Then
  \begin{equation}
    \biggl\langle u\biggm|\sum_{i=1}^nF(e_i,v)e_i\biggr\rangle
    =F\biggl(\sum_{i=1}^n\langle e_i|u\rangle e_i,v\biggr).
  \end{equation}
\end{lemma}
\begin{proof}
  $\nabla^2 f(x,y)=\frac{\partial^2f}{\partial x^2}+
   \frac{\partial^2f}{\partial y^2}$.
\end{proof}
\begin{corollary}
  This is a corollary.
\end{corollary}
\begin{remark}
  This is a remark.
\end{remark}

\chapter{Floats and references}
\begin{figure}
  \begin{center}
    %% PNG and JPG images can be inserted into the document as well,
    %% but their resolution needs to be adequate. The minimum is
    %% about 100 pixels per 1 centimeter or 300 pixels per 1 inch.
    %% That means that a JPG or PNG image typeset at 4 × 4 cm should
    %% be 400 × 400 px large at the bare minimum.
    %%
    %% The optimum is about 250 pixels per 1 centimeter or 600
    %% pixels per 1 inch. That means that a JPG or PNG image typeset
    %% at 4 × 4 cm should be 1000 × 1000 px large or larger.
    \includegraphics[width=6.3cm]{fithesis/logo/mu/fithesis-base-english-color}
  \end{center}
  \caption{The logo of \acrlong{MU} at 6.3\,cm}
  \label{fig:mulogo1}
\end{figure}

\begin{figure}
  \begin{center}
    \begin{minipage}{.5\textwidth}
      \includegraphics[width=\textwidth]{fithesis/logo/mu/fithesis-base-english-color}
    \end{minipage}
    \hfill  % Fill the horizontal space between the images
    \begin{minipage}{.33\textwidth}
      \includegraphics[width=\textwidth]{fithesis/logo/mu/fithesis-base-english-color} \\[1em]
      \includegraphics[width=\textwidth]{fithesis/logo/mu/fithesis-base-english-color}
    \end{minipage}
  \end{center}
\caption{The logo of \acrlong{MU} at $\frac12$ and
    $\frac13$ of text width}
  \label{fig:mulogo2}
\end{figure}

\begin{table}
  \begin{tabularx}{\textwidth}{lllX}
    \toprule
    Day & Min Temp & Max Temp & Summary \\
    \midrule
    Monday & $13^{\circ}\mathrm{C}$ & $21^\circ\mathrm{C}$ & A
    clear day with low wind and no adverse current advisories. \\
    Tuesday & $11^{\circ}\mathrm{C}$ & $17^\circ\mathrm{C}$ & A
    trough of low pressure will come from the northwest. \\
    Wednesday & $10^{\circ}\mathrm{C}$ &
    $21^\circ\mathrm{C}$ & Rain will spread to all parts during the
    morning. \\
    \bottomrule
  \end{tabularx}
  \caption{A weather forecast}
  \label{tab:weather}
\end{table}

The logo of \gls{MU} is shown in Figure \ref{fig:mulogo1} and
Figure \ref{fig:mulogo2} at pages \pageref{fig:mulogo1} and
\pageref{fig:mulogo2}. The weather forecast is shown in Table
\ref{tab:weather} at page \pageref{tab:weather}. The following
chapter is Chapter \ref{chap:matheq} and starts at page
\pageref{chap:matheq}.  Items \ref{item:star1}, \ref{item:star2},
and \ref{item:star3} are starred in the following list:
\begin{compactenum}
  \item some text
  \item some other text
  \item $\star$ \label{item:star1}
  \begin{compactenum}
    \item some text
    \item $\star$ \label{item:star2}
    \item some other text
    \begin{compactenum}
      \item some text
      \item some other text
      \item yet another piece of text
      \item $\star$ \label{item:star3}
    \end{compactenum}
    \item yet another piece of text
  \end{compactenum}
  \item yet another piece of text
\end{compactenum}
If your reference points to a place that has not yet been typeset,
the \verb"\ref" command will expand to \textbf{??} during the first
run of
\texttt{pdflatex \jobname.tex}
and a second run is going to be needed for the references to
resolve. With online services -- such as \Gls{Overleaf} -- this is
performed automatically.

\chapter{Mathematical equations}
\label{chap:matheq}
\TeX{} comes pre-packed with the ability to typeset inline
equations, such as $\mathrm{e}^{ix}=\cos x+i\sin x$, and display
equations, such as \[
  \mathbf{A}^{-1} = \begin{bmatrix}
  a & b \\ c & d \\
  \end{bmatrix}^{-1} =
  \frac{1}{\det(\mathbf{A})} \begin{bmatrix}
  \,\,\,d & \!\!-b \\ -c & \,a \\
  \end{bmatrix} =
  \frac{1}{ad - bc} \begin{bmatrix}
  \,\,\,d & \!\!-b \\ -c & \,a \\
  \end{bmatrix}.
\] \LaTeX{} defines the automatically numbered \texttt{equation}
environment:
\begin{equation}
  \gamma Px = PAx = PAP^{-1}Px.
\end{equation}
The package \textsf{amsmath} provides several additional
environments that can be used to typeset complex equations:
\begin{enumerate}
  \item An equation can be spread over multiple lines using the
    \texttt{multline} environment:
    \begin{multline}
      a + b + c + d + e + f + b + c + d + e + f + b + c + d + e +
f \\
      + f + g + h + i + j + k + l + m + n + o + p + q
    \end{multline}

  \item Several aligned equations can be typeset using the
    \texttt{align} environment:
    \begin{align}
              a + b &= c + d     \\
                  u &= v + w + x \\[1ex]
      i + j + k + l &= m
    \end{align}

  \item The \texttt{alignat} environment is similar to
    \texttt{align}, but it doesn't insert horizontal spaces between
    the individual columns:
    \begin{alignat}{2}
      a + b + c &+ d       &   &= 0 \\
              e &+ f + g   &   &= 5
    \end{alignat}

  \item Much like chapter, sections, tables, figures, or list
    items, equations -- such as \eqref{eq:first} and
    \eqref{eq:mine} -- can also be labeled and referenced:
    \begin{alignat}{4}
      b_{11}x_1 &+ b_{12}x_2  &  &+ b_{13}x_3  &  &             &
        &= y_1,                   \label{eq:first} \\
      b_{21}x_1 &+ b_{22}x_2  &  &             &  &+ b_{24}x_4  &
        &= y_2. \tag{My equation} \label{eq:mine}
    \end{alignat}

  \item The \texttt{gather} environment makes it possible to
    typeset several equations without any alignment:
    \begin{gather}
      \psi = \psi\psi, \\
      \eta = \eta\eta\eta\eta\eta\eta, \\
      \theta = \theta.
    \end{gather}

  \item Several cases can be typeset using the \texttt{cases}
    environment:
    \begin{equation}
      |y| = \begin{cases}
        \phantom-y & \text{if }z\geq0, \\
                -y & \text{otherwise}.
      \end{cases}
    \end{equation}
\end{enumerate}
For the complete list of environments and commands, consult the
\textsf{amsmath} package manual\footnote{
  See \url{http://mirrors.ctan.org/macros/latex/required/amsmath/amsldoc.pdf}.
  The \texttt{\textbackslash url} command is provided by the
  package \textsf{url}.
}.

\chapter{\textnormal{We \textsf{have} \texttt{several} \textsc{fonts}
  \textit{at} \textbf{disposal}}}
The serified roman font is used for the main body of the text.
\textit{Italics are typically used to denote emphasis or
quotations.} \texttt{The teletype font is typically used for source
code listings.} The \textbf{bold}, \textsc{small-caps} and
\textsf{sans-serif} variants of the base roman font can be used to
denote specific types of information.

\tiny We \scriptsize can \footnotesize also \small change \normalsize
the \large font \Large size, \LARGE although \huge it \Huge
is \huge usually \LARGE not \Large necessary.\normalsize

A wide variety of mathematical fonts is also available, such as: \[
  \mathrm{ABC}, \mathcal{ABC}, \mathbf{ABC}, \mathsf{ABC},
  \mathit{ABC}, \mathtt{ABC}
\] By loading the \textsf{amsfonts} packages, several additional
fonts will become available: \[
  \mathfrak{ABC}, \mathbb{ABC}
\] Many other mathematical fonts are available\footnote{
  See \url{http://tex.stackexchange.com/a/58124/70941}.
}.

\chapter{Inserting the bibliography}
After linking a bibliography data\-base files to the document using
the \verb"\"\texttt{thesis\discretionary{-}{}{}setup\{bib\discretionary{=}{=}{=}%
\{\textit{file1},\textit{file2},\,\ldots\,\}\}} command, you can
start citing the entries. This is just dummy text
\parencite{borgman03} lightly sprinkled with citations
\parencite[p.~123]{greenberg98}. Several sources can be cited at
once: \cite{borgman03,greenberg98,thanh01}.
\citetitle{greenberg98} was written by \citeauthor{greenberg98} in
\citeyear{greenberg98}. We can also produce \textcite{greenberg98}%
\ or %% Let us define a compound command:
\def\citeauthoryear#1{(\textcite{#1},~\citeyear{#1})}%
\citeauthoryear{greenberg98}%
. The full bibliographic citation is:
\emph{\fullcite{greenberg98}}. We can easily insert a bibliographic
citation into the footnote\footfullcite{greenberg98}.

The \verb"\nocite" command will not generate any
output\nocite{muni}, but it will insert its arguments into
the bibliography. The \verb"\nocite{*}" command will insert all the
records in the bibliography database file into the bibliography.
Try uncommenting the command
%% \nocite{*}
and watch the bibliography section come apart at the seams.

When typesetting the document for the first time, citing a
\texttt{work} will expand to [\textbf{work}] and the
\verb"\printbibliography" command will produce no output. It is now
necessary to generate the bibliography by running \texttt{biber
\jobname.bcf} from the command line and then by typesetting the
document again twice. During the first run, the bibliography
section and the citations will be typeset, and in the second run,
the bibliography section will appear in the table of contents.

The \texttt{biber} command needs to be executed from within the
directory, where the \LaTeX\ source file is located. In Windows,
the command line can be opened in a directory by holding down the
\textsf{Shift} key and by clicking the right mouse button while
hovering the cursor over a directory.  Select the \textsf{Open
Command Window Here} option in the context menu that opens shortly
afterwards.

With online services -- such as Overleaf -- or when using an
automatic tool -- such as \LaTeX MK -- all commands are executed
automatically. When you omit the \verb"\printbibliography" command,
its location will be decided by the template.

  \printbibliography[heading=bibintoc] %% Print the bibliography.

\chapter{Inserting the index}
After using the \verb"\makeindex" macro and loading the
\texttt{makeidx} package that provides additional indexing
commands, index entries can be created by issuing the \verb"\index"
command. \index{dummy text|(}It is possible to create ranged index
entries, which will encompass a span of text.\index{dummy text|)}
To insert complex typographic material -- such as $\alpha$
\index{alpha@$\alpha$} or \TeX{} \index{TeX@\TeX} --
into the index, you need to specify a text string, which will
determine how the entry will be sorted. It is also possible to
create hierarchal entries. \index{vehicles!trucks}
\index{vehicles!speed cars}

After typesetting the document, it is necessary to generate the
index by running
\begin{center}%
  \texttt{texindy -I latex -C utf8 -L }$\langle$\textit{locale}%
  $\rangle$\texttt{ \jobname.idx}
\end{center}
from the command line, where $\langle$\textit{locale}$\rangle$
corresponds to the main locale of your thesis -- such as
\texttt{english}, and then typesetting the document again.

The \texttt{texindy} command needs to be executed from within the
directory, where the \LaTeX\ source file is located. In Windows,
the command line can be opened in a directory by holding down the
\textsf{Shift} key and by clicking the right mouse button while
hovering the cursor over a directory. Select the \textsf{Open Command
Window Here} option in the context menu that opens shortly
afterwards.

With online services -- such as Overleaf -- the commands are
executed automatically, although the locale may be erroneously
detected, or the \texttt{makeindex} tool (which is only able to
sort entries that contain digits and letters of the English
alphabet) may be used instead of \texttt{texindy}. In either case,
the index will be ill-sorted.

  \makeatletter\thesis@blocks@clear\makeatother
  \phantomsection %% Print the index and insert it into the
  \addcontentsline{toc}{chapter}{\indexname} %% table of contents.
  \printindex

\appendix %% Start the appendices.
\chapter{An appendix}
Here you can insert the appendices of your thesis.

\end{document}
